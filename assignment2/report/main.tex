\documentclass{article}
\usepackage{amsmath}
\usepackage{amssymb}
\usepackage{listings}
\lstdefinelanguage{cuda}{language=C++,morekeywords={__global__,__device__,__host__,__shared__}}
% A listings language definition for Futhark.

\lstdefinelanguage{futhark}
{
  % list of keywords
  morekeywords={
    do,
    else,
    for,
    if,
    in,
    include,
    let,
    loop,
    then,
    type,
    val,
    while,
    with,
    module,
    def,
    entry,
    local,
    open,
    import,
    assert,
    match,
    case,
  },
  sensitive=true, % Keywords are case sensitive.
  morecomment=[l]{--}, % l is for line comment.
  morestring=[b]" % Strings are enclosed in double quotes.
}

\usepackage{graphicx}
\begin{document}

\section{Task 1}

\section{Task 2}



your one-line replacement;
uint32_t loc_ind = i * blockDim.x + threadIdx.x;

briefly explain why your replacement ensures coalesced access to global memory;

explain to what extent your one-line replacement has affected the performance, i.e., which tests and by what factor.



\section{Task 3}


template<class OP>
__device__ inline typename OP::RedElTp
scanIncBlock(volatile typename OP::RedElTp* ptr, const unsigned int idx) {
    const unsigned int lane    = idx & (WARP - 1);
    const unsigned int warpid  = idx >> lgWARP;
    const unsigned int n_warps = (blockDim.x + WARP - 1) >> lgWARP; // Total number of warps

    // 1. Perform scan at warp level.
    typename OP::RedElTp res = scanIncWarp<OP>(ptr, idx);
    __syncthreads();

    // 2. Place the end-of-warp results into a separate location in shared memory.
    if (lane == (WARP - 1)) {
        ptr[blockDim.x + warpid] = OP::remVolatile(ptr[idx]);
    }
    __syncthreads();

    // 3. Let the first warp scan the per-warp sums.
    if (warpid == 0 && idx < n_warps) {
        scanIncWarp<OP>(ptr + blockDim.x, idx);
    }
    __syncthreads();

    // 4. Accumulate results from the previous step.
    if (warpid > 0) {
        res = OP::apply(ptr[blockDim.x + warpid - 1], res);
    }

    return res;
}

\section{Task 4}


\end{document}
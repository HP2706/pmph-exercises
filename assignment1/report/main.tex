\documentclass{article}
\usepackage{amsmath}
\usepackage{amssymb}
\usepackage{listings}
\lstdefinelanguage{cuda}{language=C++,morekeywords={__global__,__device__,__host__,__shared__}}
% A listings language definition for Futhark.

\lstdefinelanguage{futhark}
{
  % list of keywords
  morekeywords={
    do,
    else,
    for,
    if,
    in,
    include,
    let,
    loop,
    then,
    type,
    val,
    while,
    with,
    module,
    def,
    entry,
    local,
    open,
    import,
    assert,
    match,
    case,
  },
  sensitive=true, % Keywords are case sensitive.
  morecomment=[l]{--}, % l is for line comment.
  morestring=[b]" % Strings are enclosed in double quotes.
}

\usepackage{graphicx}
\begin{document}
\section{Task 1 (2 pts)}

\subsection{Task 1.a) Prove that a list-homomorphism induces a monoid structure (1pt)}

\subsection*{Solution:}

\begin{enumerate}
    \item \textbf{Associativity:}

    By definition of $\text{Img}$, for any $x, y, z$ in $\text{Img}(h)$, there exist $a$, $b$, and $c$ such that $x = h(a)$, $y = h(b)$, and $z = h(c)$. 

    We now have:

    \begin{align*}
    (x \circ y) \circ z &= ((h(a) \circ h(b)) \circ h(c)) \\
    &= (h(a \mathbin{++} b)) \circ h(c) \quad \text{(by definition of list homomorphism)} \\
    &= h((a \mathbin{++} b) \mathbin{++} c) \\
    &= h(a \mathbin{++} (b \mathbin{++} c)) \quad \text{(as $\mathbin{++}$ is associative)} \\
    &= h(a) \circ h(b \mathbin{++} c) \\
    &= h(a) \circ (h(b) \circ h(c)) \\
    &= x \circ (y \circ z)
    \end{align*}

    \item \textbf{Neutral element:}
    Let $e = h([])$ be the neutral element. 
    For any $x$ in $\text{Img}(h)$, there exists an $a$ in $A$ such that $h(a) = x$.

    \begin{align*}
    e \circ x &= h([]) \circ h(a) = h([] \mathbin{++} [a]) = h([a]) = x \\
    x \circ e &= h(a) \circ h([]) = h([a] \mathbin{++} []) = h([a]) = x
    \end{align*}

    We also need to show that exactly one such identity element exists.
    Assume there exist two neutral elements $e$ and $e'$.

    \begin{align*}
    e \circ e' &= e \quad \text{(by definition of neutral element)} \\
    e \circ e' &= e' \quad \text{(by definition of neutral element)} \\
    e &= e' \quad \text{(by the previous equalities)}
    \end{align*}

    Therefore, there is at most one neutral element.
\end{enumerate}

\subsection{Task 1.b) Prove the Optimized Map-Reduce Lemma (1pt)}

\subsection*{Solution:}

Let $id = (\text{reduce } (++) \; []) \circ \text{distr}_p$

We start by applying the identity:

\begin{align*}
&(\text{reduce } (+) \; 0) \circ (\text{map } f) \\
&= (\text{reduce } (+) \; 0) \circ (\text{map } f) \circ  (\text{reduce } (++) \; []) \circ \text{distr}_p
\end{align*}

Now, we apply the three lemmas in sequence:

\begin{enumerate}
    \item Apply lemma 2: $(\text{map } f) \circ (\text{reduce } (++)\; []) \equiv (\text{reduce } (++)\; []) \circ (\text{map } (\text{map } f))$
    \begin{align*}
    &= (\text{reduce } (+) \; 0) \circ (\text{reduce } (++) \; []) \circ (\text{map } (\text{map } f)) \circ \text{distr}_p
    \end{align*}

    \item Apply lemma 3: $(\text{reduce } \odot\; e_\odot) \circ (\text{reduce } (++)\; []) \equiv (\text{reduce } \odot\; e_\odot) \circ (\text{map } (\text{reduce } \odot\; e_\odot))$
    
    Here, we apply the lemma with $\odot = +$ and $e_\odot = 0$. The lemma transforms:
    
    $(\text{reduce } (+) \; 0) \circ (\text{reduce } (++) \; [])$
    
    into:
    
    $(\text{reduce } (+) \; 0) \circ (\text{map } (\text{reduce } (+) \; 0))$
    
    Thus, we get:
    \begin{align*}
    &= (\text{reduce } (+) \; 0) \circ (\text{map } (\text{reduce } (+) \; 0)) \circ (\text{map } (\text{map } f)) \circ \text{distr}_p
    \end{align*}

    \item Apply lemma 1: $(\text{map } f) \circ (\text{map } g) \equiv \text{map}(f \circ g)$
    \begin{align*}
    &= (\text{reduce } (+) \; 0) \circ (\text{map } ((\text{reduce } (+) \; 0) \circ (\text{map } f))) \circ \text{distr}_p
    \end{align*}
\end{enumerate}

This completes the proof of the Optimized Map-Reduce Lemma.

\section{Task 2: Longest Satisfying Segment (LSS) Problem (3pts)}
    \subsection*{Solution:}

    \subsubsection*{1. LSSP Operator Implementation}
    \begin{lstlisting}[language=Haskell]
    let segments_connect = x_len == 0 || y_len == 0 || pred2 x_last y_first

    let new_lss = max (max x_lss y_lss) (if segments_connect then x_lcs + y_lis else 0)

    let new_lis = if segments_connect && x_lis == x_len then x_lis + y_lis else x_lis
    let new_lcs = if segments_connect && y_lcs == y_len then x_lcs + y_lcs else y_lcs

    let new_len = x_len + y_len

    let new_first = if x_len == 0 then y_first else x_first
    let new_last  = if y_len == 0 then x_last else y_last
    \end{lstlisting}

    \subsubsection*{2. Inline Tests}
    The following inline tests were added to validate the program:

    \begin{lstlisting}[language=Futhark]
    -- Small dataset for sorted
    -- ==
    -- compiled input {
    --    [1, -2, -2, 0, 0, 0, 0, 0, 3, 4, -6, 1]
    -- }  
    -- output { 
    --    9
    -- }
    -- compiled input {
    --     [1, 2, 3, 4, 5, 6, 7, 8, 9, 10]
    -- }
    -- output {
    --     10
    -- }
    -- compiled input {
    --     [10, 9, 8, 7, 6, 5, 4, 3, 2, 1]
    -- }
    -- output {
    --     1
    -- }

    -- Small dataset for same
    -- ==
    -- compiled input {
    --    [1, -2i32, -2i32, 2i32, 0i32, 0i32, 0i32, 3i32, 4i32, -6i32, 1i32]
    -- }
    -- output {
    --    3i32
    -- }
    -- compiled input {
    --     [1i32, 0i32, 3i32, 3i32, 3i32, 3i32, 6i32, 7i32, 8i32, 9i32, 10i32]
    -- }
    -- output {
    --     4i32
    -- }
    -- compiled input {
    --     [1i32, 0i32, 1i32, 0i32, 1i32, 0i32, 1i32, 0i32, 1i32, 1i32]
    -- }
    -- output {
    --     2i32
    -- }

    -- Small dataset for zeros
    -- ==
    -- entry: main
    -- 
    -- input { [0i32, 0, 0, 0, 0, 0, 0, 0, 0, 0, 1] }
    -- output { 10i32 }
    --
    -- input { [1i32, -2, -2, 0, 0, 0, 3, 4, -6] }
    -- output { 3i32 }
    --
    -- input { [0i32, 1, 0, 0, 3, 0, 0, 4, 0] }
    -- output { 2i32 }
    --
    -- input { [0i32, 1, 0, 0, 1] }
    -- output { 2i32 }
    \end{lstlisting}

    \subsubsection*{3. Performance Comparison}

    Runtimes and speedups:

    \begin{table}[h]
    \centering
    \begin{tabular}{|l|r|r|c|}
    \hline
    \textbf{Benchmark} & \textbf{Sequential Runtime} & \textbf{Parallel Runtime} & \textbf{Speedup} \\
    \hline
    lssp-zeros & 286.3 & 42.1 & 6.8x \\
    lssp-sorted & 473.3 & 41.6 & 11.4x \\
    lssp-same & 181.4 & 42.0 & 4.3x \\
    \hline
    \end{tabular}
    \caption{Runtime comparison and speedup for LSSP benchmarks}
    \label{tab:lssp-benchmarks}
    \end{table}

    \textit{Note: The values shown are placeholder values. Please replace them with the actual measured values for each benchmark.}

\end{itemize}

\section{Task 3: CUDA Exercise (3pts)}

\subsubsection{Kernel Code}

\begin{lstlisting}[language=cuda]
__global__ void cuda_map(float* X, float* Y, int n) {
    const unsigned int i = blockIdx.x * blockDim.x + threadIdx.x;
    if (i < n) {
        float x = X[i]; // Load the input element
        float temp = __fdividef(x, x - 2.3f); 
        Y[i] = temp * temp * temp; // Avoid using pow()
    }
}
\end{lstlisting}

\subsubsection{Grid and Block Size Computation}

\begin{lstlisting}[language=cuda]
int BlockSize = 1024;
int blocksPerGrid = (N + BlockSize - 1) / BlockSize;
\end{lstlisting}

\subsubsection{Kernel Invocation}

\begin{lstlisting}[language=cuda]
cuda_map<<<blocksPerGrid, BlockSize>>>(d_in, d_out, N);
\end{lstlisting}

\subsubsection{Validation}

\begin{lstlisting}[language=cuda]
validate<float>(h_out, h_out_seq, N, 0.000001);
\end{lstlisting}

\subsection{Memory Throughput Analysis}

\begin{itemize}
    \item Peak bandwidth at initial length (753,411): 151 GB/s
    \item Maximum bandwidth achieved: 1,097.57 GB/s
    \item Array length for maximal throughput: $2^{25} = 33,554,432$
    \item Peak memory bandwidth of GPU hardware (A100 40GB): 1,555 GB/s
\end{itemize}

\subsection{Conclusion}

The implementation fulfills all criteria of the task. The CUDA version achieves a significant speedup over the CPU version and approaches the theoretical peak memory bandwidth of the GPU hardware when using larger array sizes.

\section{Task 4: Flat Sparse-Matrix Vector Multiplication in Futhark (2pts)}


\subsection{Solution}

\subsubsection{Implementation}

\begin{lstlisting}[language=futhark]
-- entry: main
-- input {
--   [0i64, 1i64, 0i64, 1i64, 2i64, 1i64, 2i64, 3i64, 2i64, 3i64, 3i64]
--   [2.0f32, -1.0f32, -1.0f32, 2.0f32, -1.0f32, -1.0f32, 2.0f32, -1.0f32, -1.0f32, 2.0f32, 3.0f32]
--   [2i64, 3i64, 3i64, 2i64, 1i64]
--   [2.0f32, 1.0f32, 0.0f32, 3.0f32]
-- }
-- output { [3.0f32, 0.0f32, -4.0f32, 6.0f32, 9.0f32] }
-- input @ data.in
-- output @ data.out

let spMatVctMult [num_elms][vct_len][num_rows]
    (mat_val: [num_elms](i64, f32))
    (mat_shp: [num_rows]i64)
    (vct: [vct_len]f32)
      : [num_rows]f32 =

    -- Compute the flag array using the mkFlagArray function from the lecture notes p. 48
    -- The flag array gives us a boolean array where true means that the element is the last in its row.
    let flag_arr = mkFlagArray mat_shp false (replicate num_rows true) 

    -- Cast the flag array to the type of the products array
    let typed_flag_arr = flag_arr :> [num_elms]bool

    -- Map across the list of tuples, index into vec with 
    -- the first tuple element and multiply by second tuple element
    -- This gives us the product of the matrix and the vector
    let products = map (\(ind, value) -> 
        value * vct[ind]
    ) mat_val

    -- Use the segmented scan to sum over the products within each row
    let scan_res = sgmSumF32 typed_flag_arr products

    -- Get the indices of the last element of each row by doing a scan over the row shapes
    -- By summing we can compute the global index of the last element of each row
    let last_indices = scan (+) 0 mat_shp

    -- Extract the last element of each segmented sum
    let row_sums = map (
        \i -> if i == 0 then 
            scan_res[i]
        else 
            scan_res[i - 1]
    ) last_indices

    in row_sums
\end{lstlisting}

\subsubsection{Benchmarking}

We benchmark on a dataset generated with:

\begin{verbatim}
futhark dataset --i64-bounds=0:9999 -g [1000000]i64 --f32-bounds=-7.0:7.0 -g [1000000]f32 --i64-bounds=100:100 -g [10000]i64 --f32-bounds=-10.0:10.0 -g [10000]f32 > data.in
\end{verbatim}

The sequential version takes 1676µs while the flat version takes 202µs, which is a speedup of approximately 8x.

\end{document}